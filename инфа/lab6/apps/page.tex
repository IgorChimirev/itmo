\setcounter{page}{44}
\newgeometry{top=0cm, bottom=1.8cm, left=3.3cm, right=3.5cm} 

\begin{multicols*}{2}

\vspace{0.5cm}
\noindent
для L неотрицательны, найденное решение — оптимальное;
в противном
случае процесс выбора базисных переменных и улучшения решения про-должается.
\newline\indentОписанный алгоритм и называется  \textit{симплекс-методом}. (Обратили ли Вы, читатель, внимание, что в нашем изложении алгоритма оставлена одна «дырка»: одна возможность не рассмотрена?)
\newline
\newline\textbf{Два примера}
\newlineРассмотрим сначала обычную школьную задачу. 
\newline\indentЗадача 1. 
\textit{Три школьника хотят добраться до лесного озера, распо-ложенного в 20 км от дома. У них есть один двухместный молед, на котором можно ехать со скоростью 36 км/час. Пешком каждый школьник может идти со скоростью 4 км/час.
Как организовать движение, чтобы всем троим быстрее добраться до озера? Каково наименьшее время, за которое это можно сделать?}
\newline\indentОчевидно, что многократная смена пассажиров на мопеде не сможет дать никакой выгоды во времени по сравнению с однократной. Поэтому составим такой план организации движения: в начальный момент времени из дома Д одновременно выезжают на мопеде два школьника и выходит пешком третий школьник. В промежуточной точке Т водитель мопеда высаживает своего спутника, который до озера О идет дальше пешком; мопед же возвращается за третьим школьником, встречает его в точке К и отвозит к озеру (на рисунке 1 красным цветом обозначено движение мопеда, синим — движение пешеходов).
Наш план организации движения становится конкретным при фиксировании расстояння \textit{|ДТ| = x}. Нам надо найти х, при котором время і прибытия в точку О последнего из школьников будет минимальным.
\newline\indentОбозначим через t$_{1}$ время, затраченное на «переход» из д в О «высаженным» школьником, через t$_{2}$~---\penalty10000
\columnbreak


\begin{tikzpicture}
    % Оси
    \draw[->, thick] (-1,0) -- (9,0) node[right] {};
    \draw[->, thick] (0,-1) -- (0,5) node[above] {};
    
    % Красная ломаная линия
    \draw[thick, red] (0,0) -- (2,3) -- (4,1) -- (6,4);
    
    % Голубые линии
    \draw[thick, cyan] (0,0) -- (4,1);
    \draw[thick, cyan] (2,3) -- (8,4);
    
    % Пунктирные линии
    \draw[thick, dash pattern=on 3pt off 3pt]
        (6,0) -- (6,4)
        (8,0) -- (8,4)
        (0,4) -- (8,4)
        (2,0) -- (2,3)
        (4,0) -- (4,1)
        (0,3) -- (2,3)
        (0,1) -- (4,1);
    
    % Метки
    \node at (-0.5,4) {O};
    \node at (-0.5,3) {T};
    \node at (-0.5,1) {K};
    \node at (-0.5,0.25) {Д};
    
    % Отметки на оси x
    \foreach \x in {0,2,4,6,8}
        \draw[shift={(\x,0)}] (0pt,2pt) -- (0pt,-2pt);
\end{tikzpicture}
\noindent
время, затраченное водителем мопеда. Очевидно, \[ t_1 = \frac{x}{36} + \frac{20 - x}{4}.\]

\noindent Положим $|KT| = y$. Тогда \[ \frac{x-y}{4} = \frac{x+y}{36}, \: \text{откуда} \: y = \frac{4}{5}x. \: \text{Очевидно}, \]

\vspace{1em}
\noindent\[ t_2 = \frac{20 + 2y}{36}. \: \text{Значит}, \: t_2 = \frac{5}{9} + \frac{2}{45}x.\]

\vspace{1em}
\noindent По смыслу наших обозначений 
\noindent\newline
\indent\indent$t = \max(t_1, t_2)$.
\noindent
\newlineТаким образом, $t \geq t_1$ и $t \geq t_2$. Если 
«высаженный» школьник прибыл в 
$O$ раньше, чем мопед, то $t_1 < t_2$ и $t = t_2$.
В противном случае $t_1 \geq t_2$ и  $t = t_1$.\noindentОтметим ещё, $t \geq 0$.

\indentНам надо найти такое $x$, при котором функция $t = \max(t_1, t_2)$ принимает наименьшее значение. Кроме

\noindent
того, нам надо найти значение функции $\max(t_1, t_2)$ при этом $x$, т.е есть число\noindent$\min(\max(t_1, t_2)$).
Из рисунка 2 (на нём красным цветом нарисована график функции $t_1$(x), чёрным — функции $t_2$(x), синим\noindentфункции $\max(t_1, t_2)$) сразу видно, что искомое определяется из уравнения \noindent$t_1$ = $t_2$,

Тем не менее, чтобы проиллюстрировать общий метод, поставим и 
решим данную задачу как задачу линейного программирования. Переформулируем её так: найти наименьшее $t$, для которого одновременно
$t \geq t_1$ и $t \geq t_2$. Итак, нам надо найти неотрицательные значения переменных $x$ и $t$, которые удовлетворяли бы системе неравенств

\begin{equation*}
  \begin{cases*}
    t \geq \frac{x}{36} + \frac{20 - x}{4},
   \\
    t \geq \frac{5}{9} + \frac{2}{45}x. 
  \end{cases*}
  \tag{8}
\end{equation*}
\newgeometry{left=1.5cm, right=1.5cm, top=1.5cm, bottom=1.5cm}